%!TEX root = ../thesis.tex
%*******************************************************************************
%****************************** Eighth Chapter *********************************
%*******************************************************************************
\chapter{おわりに}
\graphicspath{{Chapter8/Figs/}}


本論文では,GitHubのイシューをプログラミング演習問題へと転用するシステムであるRealCodeについて述べた.
TA経験者および学生によるシステム評価を通して,イシューをプログラミング演習問題へと転用することで,既存の学習環境にはない独自の演習問題を提供可能であることが明らかとなった.

今後の課題として,考察で述べたシステムのさらなる改良,および学習効果の評価のほか,演習問題を提供するインタフェースの改良が挙げられる.
例えば,よりインタラクティブ性のあるもの(例えば,解答に時間がかかっている場合にはヒントを与えるなど)などを取り入れることが考えられる.
また,スマートフォンなどのモバイルデバイス上で学習できるようにすることで,学習者のすきま時間を利用した学習を支援するなど,新しい形のプログラミング学習を実現できると考える.
本研究で得られた知見をもとにして,RealCodeがさらに発展していくことを期待する.