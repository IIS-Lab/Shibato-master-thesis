%!TEX root = ../thesis.tex
%*******************************************************************************
%*********************************** First Chapter *****************************
%*******************************************************************************

\chapter{はじめに}  %Title of the First Chapter

\graphicspath{{Chapter1/Figs/}}

\section{背景}
プログラミング演習問題はソフトウェアに関するスキルを高めるために広く利用されている手段である.
プログラミングの教科書では学習者が学んだことを復習するために演習問題が使用されている.
加えて,演習問題を通じて学習者はプログラミングのさらなる知識やスキルを習得することができる.
したがって,プログラミング教育において演習問題の重要性は広く認知されている~\cite{ericsson1993role,Winslow:1996:PPP:234867.234872}.


しかし,学校や教科書における演習問題の多くが,アルゴリズムやデータ構造といった基礎的知識を重視しており,実際のソフトウェア開発に関する知識やスキルを対象することは限られている~\cite{Allen:2003:PPC:611892.611940}.
さらに,新しいプログラミング言語やライブラリに関する演習問題を手動で作り出すのは,教師にとって大きな負担となり,生成される演習問題の数にも限りがある.
したがって,プログラミング演習問題で扱われる内容と実践的なソフトウェア開発の間には隔たりが生じている.
このため,実際のソフトウェア開発に即したプログラミング演習問題をより効率的に生成するシステムがあれば,次々と拡張されていく様々なプログラミング言語を学習することの支援につながる.

そこで本研究では,実際のソフトウェア開発に関係したプログラミング演習問題を学習者に提供するために,GitHub\footnote{\url{https://github.com}}のコード変更データを転用することで演習問題を生成するRealCodeというシステムを実装した.
RealCodeにおいてプログラミング演習問題を提供するため,我々は機械学習の手法(現在のシステムでは決定木)を用いて,演習問題として適用可能なGitHub上のイシューと関連するプルリクエストのコード変更を抽出した.
これらをそれぞれ演習問題の問題文とその解答コード例として,3つのインタフェース(単語帳形式,選択式,及び穴埋め問題)を通して提供することで,実際のソフトウェア開発で行われたコード変更を学習素材として利用可能としている.
TA経験者および学生によるRealCodeのシステム評価を行った結果,GitHubのイシューをプログラミング演習問題へと転用することで,既存の学習環境にはない独自の演習問題が提供可能となることが明らかとなった.


\section{貢献}
本研究の目的はGitHub上にて公開されているイシューとプルリクエストをプログラミング演習問題へと転用することの実現可能性を示し,実際に生成された演習問題の妥当性,および教科書等との問題の性質の差を評価することにある.
本研究がソフトウェア工学とヒューマン・コンピュータ・インタラクションの分野にもたらす貢献は以下の通りである.

\begin{itemize}
\item 演習問題として適用可能なGitHub上のイシューと関連するプルリクエストのコード変更を抽出する手法
\item RealCode上で提供される演習問題の妥当性の評価
\item RealCode上で提供される演習問題と教科書等における演習問題の性質の差に関する評価
\end{itemize}