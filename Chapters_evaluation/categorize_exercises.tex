%#
%# Lab study
%# https://github.com/IIS-Lab/ShibatoM/wiki/Analysis-of-lab-study をまとめる.
%############################

\chapter{RealCodeが出題する演習問題の分類評価}
\label{section:lab-study}
\graphicspath{{Chapters_evaluation/Figs/}}


\ref{section:ta_evaluation},\ref{section:interview-study}章にて述べた評価実験により,イシューをプログラミング演習問題へと転用する事で,既存の学習環境にはない独自のプログラミング演習問題を提供可能である事が明らかとなった.
しかし,RealCodeの演習問題から何を学べるのかは明らかとなっていない.
そこで我々は,RealCodeから何を学べるのかを理解するために,著者によるRealCodeの演習問題の評価を行った.
本章では,この評価実験の詳細と結果について述べる.


\section{実験方法}

現在のRealCodeのプロトタイプに存在する \xxx 件の演習問題に対し,図\ref{fig:lab-study}に示す本実験用のインターフェースから,著者は以下の質問に回答した.

\begin{itemize}
  \item[Q1-1.] Is this a good exercise to you? (1: Disagree -- 3: Agree)
  \item[Q1-2.] Why you selected ``Disagree''?
  \item[Q2.] What did you learn from this exercise?
  \begin{itemize}
  	  \item 文法
   	  \item アルゴリズム
      \item ライブラリやAPIの使い方
      \item 自分が経験した・見たことのあるbugの解決方法
      \item ソフトウェアのテスト
      \item デザイン
      \item Web
      \item DB
      \item 例外処理
      \item メンテナンス
      \item その他学んだこと(記述式)
  \end{itemize}
\end{itemize}

\begin{figure}[H]
 \centering
  \includegraphics[width=1.0\columnwidth]{20181218-lab-study-interface-all.png}
  \caption{評価に用いたインターフェース.}
  \label{fig:lab-study}
\end{figure}

\section{実験結果と考察}

Q1-1に対して,40.8\%の演習問題が``Agree'',17.9\%が``Neither'',41.3\%が``Disagree''として回答された.
``Disagree''とラベル付けされた演習問題に対するQ1-2の回答として,1)コード変更がわかりづらい,2)リファクタリングのみが行われている,3)著書が知らなかったライブラリのAPIが多く使用されている,の3つの要因が多く挙げられた.

コード変更がわかりづらい要因として,コード変更量が数行程度といった断片的であることや,既存の関数の一部の変更といった局所的であることが明らかとなった.
また,使用されている変数がリポジトリ特有である場合も,コード変更がわかりづらいと判断した.
これらのような演習問題をデータセットから除外する手法の一つとして,演習問題のコード変更と全体のソースコードの構文解析を行い,外部変数の使用割合や文法的コード変更の分類を特徴量として決定木に与えることが考えられる.
一方で,現在のRealCodeのプロトタイプは様々なプログラミング言語に対応するために意図的に構文解析を導入していない.
RealCodeの演習問題において特に頻繁に使用されているプログラミング言語の構文解析を実装することで,演習問題数を大きく減らすことなく,コード変更がわかりづらい演習問題を除外できると考えられる.

リファクタリングのみが行われている演習問題からは,変数名やインデントの変更などのみから構成されているため,プログラミングに関する知識を得ることができないと判断した.
コード変更がリファクタリングであるかどうかを判定するためには,構文解析を使用した手法と,コード変更の繰り返しを検出する手法が考えられる.
構文解析を用いてコード変更が構造的に変更されているかどうかを判定することで,構造的変更を行わないリファクタリングを検出できると考えられる.
また,コード変更内に繰り返し発生する変更パターンを検出できた場合も,リファクタリングであると判断できる.
後者は構文解析を用いることなく実装できるため,様々なプログラミング言語に対応しながら実装することができる.

また,著者が知らなかったライブラリのAPIが多く使用されている演習問題では,そもそもコード変更を理解することが出来なかったため,``Disagree''と回答した.
このような演習問題の多くが,デバッグツールやコンソールアプリといった,著者にとってあまり馴染みのないリポジトリから生成された.
ユーザのプログラミング経験や知識に応じて,演習問題の生成元であるリポジトリをあらかじめ選定することで,ユーザが知らないライブラリが多く使用された演習問題を除外できる可能性があると考える.



\begin{table}[t]
  \centering
  \caption{WIP}
  \label{table:lab-study-q2-result}
  \begin{tabular}{ l | c } \Xhline{3\arrayrulewidth}
      分類 & 割合(\%) \\ \hline \hline
      例外処理 & 12.85 \\
      メンテナンス & 12.29  \\
      自分が経験した・見たことのあるbugの解決方法 & 9.5 \\
      ライブラリやAPIの使い方 & 6.7 \\
      アルゴリズム & 2.79 \\
      Web & 2.79 \\
      文法 & 1.12 \\  
      デザイン & 0.56 \\
      DB & 0.56 \\
      \Xhline{3\arrayrulewidth}
  \end{tabular}
\end{table}

次に,Q2の回答結果を表\ref{table:lab-study-q2-result}に示す.
特に回答の割合の高かった例外処理,メンテナンス,自分が経験した・見たことのあるbugの解決方法,ライブラリやAPIの使い方の4つの分類について考察を行う.


\subsection{例外処理に関連する演習問題}

\begin{figure}[t]
    \begin{tabular}{cc}
      %---- Flip card, before ---------------------------
      \begin{minipage}[t]{1.0\columnwidth}
        \centering
        \includegraphics[width=1.0\columnwidth]{20190107-lab-study-exception-exercise.png}
        \subcaption{例外処理1.}
        \label{fig:lab-study-eg-exception-1}
      \end{minipage} \\
      %---- Flip card, after --------------------------
      \begin{minipage}[t]{1.0\columnwidth}
        \centering
        \includegraphics[width=1.0\columnwidth]{20190107-lab-study-exception-exercise2.png}
        \subcaption{例外処理2.}
        \label{fig:lab-study-eg-exception-2}
      \end{minipage}
    \end{tabular}
    \caption{例外処理.WIP.}
    \label{fig:lab-study-eg-exception}
\end{figure}


例外処理に関して学ぶことができるとラベル付けされた演習問題には,文字コードのEncodeエラーやネットワークの接続エラーが発生した際の対応策に関するものが多く含まれていた.
図\ref{fig:lab-study-eg-exception}に示す演習問題の例では,Pythonのtry,catch構文を用いて例外処理に対応している.
ソフトウェアを本番稼働させるためには,起こりうる例外を想定しあらかじめ対策しておくことが重要であるが,一般的なプログラミング演習問題ではあまり扱われていない.
RealCodeの例外処理に関する演習問題を通じて,ソフトウェアを本番稼働させた際に発生しうる問題と,その対応策を学ぶことができると考えられる.


\subsection{メンテナンスに関連する演習問題}

\begin{figure}[t]
	\centering
  \includegraphics[width=1.0\columnwidth]{20190107-lab-study-maintenance-exercise.png}
  \caption{メンテナンス.WIP.}
  \label{fig:lab-study-eg-maintenance}
\end{figure}

ソフトウェアを安定的に本番稼働させ,障害発生時に迅速に対応するためには,動作中のソフトウェアのログを取得しておくことが重要である.
図\ref{fig:lab-study-eg-maintenance}に示す演習問題の例では,動作しているソフトウェアのプロセス番号をログに追加するコード変更を行なっている.
プロセス番号があらかじめ分かっていれば,ソフトウェアの動作状況を監視できると同時に,障害発生時にも速やかな対応を行うことが可能となる.
RealCodeのメンテナンスに関する演習問題から,ソフトウェアの実装だけでなく,運用していく上での知識も学習できる可能性がある.

\subsection{自分が経験した・見たことのあるbugの解決方法に関連する演習問題}

\begin{figure}[t]
 \centering
 \includegraphics[width=1.0\columnwidth]{20190107-lab-study-experience-exercise.png}
  \caption{経験.WIP.}
  \label{fig:lab-study-eg-experience}
\end{figure}

RealCodeの演習問題はGitHub上のデータから生成されており,実際によく発生するバグに対応するためのイシューやプルリクエストが多く含まれている.
図\ref{fig:lab-study-eg-experience}に示す演習問題では,データベースの名前にピリオドは使用できないという,データベースを扱う際に頻発するバグを取り扱っている.
このような内容は一般的なプログラミング学習材料では扱われておらず,RealCode特有の演習問題であると言える.


\subsection{ライブラリやAPIの使い方に関連する演習問題}

\begin{figure}[t]
	\centering
  \includegraphics[width=1.0\columnwidth]{20190107-lab-study-lib-exercise.png}
  \caption{ライブラリ.WIP.}
  \label{fig:lab-study-eg-lib}
\end{figure}

実際のソフトウェア開発では,既に機能がパッケージとして実装されたライブラリを活用することで,開発を効率的に進めている.
図\ref{fig:lab-study-eg-lib}に示す演習問題の例は,Pythonのseaborn\footnote{\url{https://github.com/mwaskom/seaborn}}というデータの可視化に使用されるライブラリにおける,グラフ表示のAPIの使用方法に関する問題である.
このように,RealCodeの演習問題を通じて,ライブラリを使用する上で発生しうる問題を学習できる可能性がある.
