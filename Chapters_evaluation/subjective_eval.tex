%!TEX root = ../thesis.tex
%*******************************************************************************
%****************************** 6th Chapter **********************************
%*******************************************************************************
\chapter{RealCodeが出題する演習問題の学習者視点から主観的評価}
\graphicspath{{Chapter6/Figs/}}

\label{section:interview-study}

\ref{section:ta_evaluation}章にて述べたRealCodeが出題する演習問題のTA経験者による独自性評価により,GitHubのイシューを転用することで実現された演習問題の特徴を明らかにすることができた.
続いて我々は,学習者がRealCodeの演習問題をどう感じるのか理解するために,プログラミングの講義の受講経験とPythonの使用経験を持つ8人の大学生(PC1--PC8, 全て男性)にインタビューを行った.
RealCodeのインターフェースの簡単な説明の後に,実験参加者に自由にRealCodeを操作してもらった.
実験参加者の主観的な印象に先入観を与えないため,RealCodeがどのようにプログラミング演習問題を生成しているのかの説明は意図的に省略した.
そして,実験の最後にRealCodeに対する印象や感想について約20分のインタビューを行った.


% \ref{section:issue-classification}章にて述べたイシューの分類により,プログラミング演習問題として活用出来ると考えられるデータセットを構築することができた.
% しかしRealCodeが提供する演習問題のユーザ・エクスペリエンスはまだ明らかとなっていない.
% そこで我々はユーザがRealCodeの演習問題をどう感じるのか理解するために,ソフトウェアの講義の受講経験がありかつPythonの使用経験がある8人の大学生(PC1--PC8, 全て男性)にインタビューを行った.
% 簡単なRealCodeのインターフェースの説明の後に,実験参加者に自由にRealCodeを操作してもらった.
% 実験参加者の主観的な印象に先入観を与えないために,我々はRealCodeがどのようにプログラミング演習問題を生成しているのかを意図的に説明しなかった.
% そして,実験の最後にRealCodeに対する印象や感想について約20分のインタビューを行った.
% 本章ではインタビューの結果について述べる.

% %Our quantitative examination on selection heuristics revealed a potential to extract appropriate issues and code diffs for exercise creation.
% %However, the user experience of programming exercises generated by RealCode is still unknown.
% We next conducted an informal qualitative study to understand how learners perceive exercises in our system.
% We recruited eight university students (PC1--PC8, all male) who had taken at least one software course and had Python programming skills.
% After we explained our interfaces, participants were asked to freely explore the system.
% We deliberately provided no explanation about how our system generated exercises to avoid potential biases in their subjective impressions.
% Our post-experimental interviews were conducted to examine how our participants described RealCode exercises.
% %We offered approximately 10 USD in local currency as a compensation. % 1000 yen.

% We transcribed the interviews and performed open-ended coding to categorize quotes.
% We faithfully translated quotes as faithfully as possible for report because our interviews were conducted in a local language.

\section*{開発にて実際に発生したバグ・トラブルの体験}
RealCodeの最も特徴的な要素は,プログラミング演習問題をGitHub上に実在するイシューおよびプルリクエストから生成していることにある.
例えばバグ修正は実践的なソフトウェア開発における一般的な作業であるが,一般的なプログラミング演習問題ではあまり扱われていない.
PC2とPC3はバグ修正に関するRealCodeの演習問題が彼らにとって新鮮な体験であると述べた.

% The most unique aspect of exercises in RealCode is that they are based on actual issues and code diffs.
% Bug fixes are a common activity in real-world software development, but existing programming exercises do not cover in general.
% PC2 and PC3 were excited when they saw a bug fix exercise because it is not what they would normally see in textbooks.
% Such exercises also encouraged our participants to consider what they would need to learn as PC1 commented.

% P2: トラブルシューティングみたいな問題は見たことないから,結構面白いかも

% P3: あんまりこういうエラーが発生してそれにどう対処すべきか,みたいな問題を経験したことがないから,なんか新鮮で面白そう.
\myquote{トラブルシューティングみたいな問題は見たことないから,結構面白いかも.}{PC2}

\myquote{あんまりこういうエラーが発生してそれにどう対処すべきか,みたいな問題を経験したことがないから,なんか新鮮で面白そう.}{PC3}
% \myquote{I have never tried exercises on how to solve an error code. It is eye-opening and intriguing to me.}{PC3}

PC1は,このようなバグ修正の演習問題は実践的な知識の学習意欲を向上させる可能性があると述べており,これはRealCodeが提供する演習問題の1つの利点であると考えられる.


% P1: 開発における問題点の解決のためのプログラミング知識を意識させられるなあと感じた
\myquote{(RealCodeの演習問題は)開発における問題点の解決のためのプログラミング知識を意識させられるなあと感じた.}{PC1}
% \myquote{[Exercises in our system] made me think of what programming knowledge I would need to solve problems in real development.}{PC1}

PC6はRealCodeの演習問題はソフトウェア開発の授業においてより適当であるという感想を述べている.
これもRealCodeが教科書等とは違った演習問題を提供できていることを示唆するコメントである.

% PC6 ultimately summarized the characteristics of exercises in RealCode by emphasizing that they would fit to development-oriented learning.

% P6: 普通の授業で使うのは少し違う気がする.授業で扱うべきは基礎だと思うし.ただ,プロジェクト型の授業とか,学ぼうというより開発系の授業なら全然いいと思う.
\myquote{普通の授業で使うのは少し違う気がする.授業で扱うべきは基礎だと思うし.ただ,プロジェクト型の授業とか,学ぼうというより開発系の授業なら全然いいと思う.}{PC6}
% \myquote{I feel [RealCode] would not work in normal classes. They should teach basics. But if you are in project-based or development-oriented classes, [RealCode] would be really good.}{PC6}

\section*{プログラミング知識の実践的な活用方法の体験}

実験参加者は講義や教科書を通じてPythonを学んだ経験があるが,Pythonの実践的な開発経験は必ずしもなく,RealCodeでの演習問題を通して,Pythonの様々な利用法を見ることが可能となっていた.

% Our participants had learning experience of Python through courses and textbooks.
% But they did not necessarily experience different applications of the language due to lack of development opportunities.
% Two participants (PC2 and PC8) commented their excitement of discovering new Python use.

% P2: (問題をみて)へーPythonってこんな低レイヤーなことできるんですね
\myquote{(問題をみて)へーPythonってこんな低レイヤーなことできるんですね.}{PC2}
% \myquote{(Seeing one exercise,) I did not know that Python can do this low-level thing.}{PC2}

% P8: 今までやったPythonって文法とデータ分析くらいで,web周りに使ってるのは初めて見たし面白い
\myquote{今までやったPythonって文法とデータ分析くらいで,web周りに使ってるのは初めて見たし面白い.}{PC8}
% \myquote{I've learned Python only about the syntax and for data analysis. This is my first time to see Python for Web development, and it makes the exercises interesting.}{PC8}

上記のような例は実際のソフトウェア開発はよく見られる可能性があるものの,教科書等では記載されていることが少なく,RealCodeが既存の教育素材とは違った視点で,演習問題を提供できていることを示す1つの例である.

また,RealCodeの演習問題は問題文に開発目的や背景知識を含んでいるものが多い.
そういった情報は開発の動機を説明しており,演習問題が実践的なソフトウェア開発を想定していることを実験参加者に実感させることも可能となっていた.

% Similarly, exercises on RealCode often contain background stories and explanations of development objectives.
% They explain motivations for revising code, and improve the perceived practicality of exercises on RealCode.

% P4: 開発の背景というか現状把握に時間がかかるけど,話にストーリーがあって面白い.(~ would be great, というセリフを見て)何かのサービスを改善したい,から問題が始まってるのは,学習のモチベーション的にいい.
\myquote{開発の背景というか現状把握に時間がかかるけど,話にストーリーがあって面白い.(``would be great'' という一文を見て)何かのサービスを改善したい,から問題が始まってるのは,学習のモチベーション的にいい.}{PC4}

% \myquote{It takes time to understand the background of an exercise. But such a story makes the exercise interesting. This problem has a story like a person wants to improve a service. That's a good motivation for learning too.}{PC4}


\section*{開発チームにおけるコミュニケーションの体験}


ソースコードの可読性はチーム開発における開発速度やソフトウェアの質そのものに大きく寄与するが,可読性に重点をおいているプログラミング演習問題は少ない.
一方,RealCodeは実際のコード変更を演習問題に転用しているため,可読性の高いコードやコード中のコメントを含んでいることが実験参加者によって指摘されていた.
そのような解答コード例は学習者の可読性に対する意識を高めることができる可能性があり,RealCodeはコードの書き方に関する教育的効果をもたらす可能性が示唆された.

% Source code readability is critical for quality control as well smooth communication among developers.
% But existing exercises do not typically emphasize readability.
% PC2 and PC5 noticed that answer keys on RealCode often contain readable code and many comments.
% These answer keys can contribute to increasing learner's awareness of code readability.

%P2: なんかコードが形式ばってないですね,わりと可読性を意識して書かれてる気がします.

% P5: コード内にコメントがたくさんあって,なんとなくこう可読性というか人に見られることを意識していていいね.
\myquote{コード内にコメントがたくさんあって,なんとなくこう可読性というか人に見られることを意識していていいね.}{PC5}

% \myquote{Answer code includes many comments. Looks like the code is intended to be seen by others, and that's good.}{PC5}


また,PC8はソフトウェア開発におけるコミュニケーション特有の用語や略語が多く含まれている事を指摘した.
実際の開発ログから演習問題を生成する事で開発特有の表現が含まれるようになるため,実践的な開発の一部を垣間見ることができることが示唆されている.

% Participants also noticed slangs and abbreviations that are commonly used in communication among developers. 
% Exercises in textbooks do not contain such terms, and our participants felt more reality by seeing them.

\myquote{英語というか,実践的な表現が学べて面白い.開発特有の表現とか.LGTMとかエラーの表現とか今まで知る機会がなかった.}{PC8}

% \myquote{It is interesting to see expressions exercises use. I did not have an opportunity to know ``LGTM (looks good to me)'' or other error-related expressions until now.}{PC8}



% Comparison to programming classes
% P1: 講義は教養的な知識が多すぎる
% P6: 授業の方がもっと根本からプログラミングを学べるけど,少しハイレイヤーというか俯瞰的に学ぶための問題としてなら結構いいと思う






\section*{各インターフェースの特徴}

8人の実験参加者のうち,7人が選択式のインターフェースが最も良いと述べた.
選択式のインターフェースでは解答方法が簡単であるため,プログラミング学習を始めたばかりの学習者に適していると考えられる.

% Seven of the participants preferred \textit{MultiChoice} the most.
% They agreed that \textit{MultiChoice} offered a good starting point for learning as answering does not necessarily require proficient knowledge.

% P1: 1番目(MCQ)は気軽に取り組みやすそう.特に俺みたいな授業でPythonやっただけの人には答えやすい
\myquote{(選択式のインターフェース)は気軽に取り組みやすそう.特に俺みたいな授業でPythonやっただけの人には答えやすい}{PC1}
% \myquote{\textit{MultiChoice} looks a good entry point to me. Especially for those who learned Python only in a class like me, it is easy to answer.}{PC1}


% Fill
3人の実験参加者ら(PC5,PC6,PC8)は,穴埋め式のインターフェースではPythonの文法を詳しく学ぶことができると指摘した.
一方でPC2は,穴埋め式のインターフェースではPythonの文法のごく一部のみが扱われるため,選択式のインターフェースの方が学習に適していると述べた.
これらのことから,学習者が学習したい内容に合わせて,インターフェースが使い分けされることが想定される.
% Three interviewees (PC5, PC6, and PC8) also liked \textit{FillBlank} because they can learn details on Python syntax.
% However, PC2 pointed out that \textit{MultiChoice} is better because requires learners to examine the whole code changes instead of only a part of it.
% P2: 穴埋めだと一つの構文の勉強だけになっちゃうから,構文は結構わかるし,全体的に読む(MCQ)問題形式の方がいい
\myquote{穴埋めだと一つの構文の勉強だけになっちゃうから,構文は結構わかるし,全体的に読む問題形式(選択式)の方がいい}{PC2}
% \myquote{The fill-in-the-blank is more for studying one particular syntax rule. But I already know syntax well, so \textit{MultiChoice} is better for me because I need to read all lines [of code changes].}{PC2}




% Flip
他のインターフェースと異なり,単語帳形式のインターフェースは実験参加者らから不評であった.
特にプログラミング初心者にとって,イシューの説明文から解答となるコードを推測することは難易度が高すぎることが示唆されていた.

% Unlike the other interfaces, \textit{FlashCard} was not popular in our study.
% PC7 stated that it is too difficult for non-experts to give any similar code to the answer key.
% P7: この解答どうりに答えれることは無いし,答え合わせがしづらくて嫌かも.上級者にはいいのかもしれないが,僕には少し難しすぎるかもしれない
\myquote{この解答どうりに答えれることは無いし,答え合わせがしづらくて嫌かも.上級者にはいいのかもしれないが,僕には少し難しすぎるかもしれない.}{PC7}
% \myquote{It is impossible to give the exact same answer to the answer code. Also hard to check if I am correct. Maybe good for advanced learners, but a little too difficult to me.}{PC7}



% \subsection*{Areas for Improvements}
% Although the interviewees found benefits of \XXX, they also pointed out several limitations.
% P4, P6 and P7 stated that it is hard to understand contexts of exercises on \XXX.
% % P4: 何を問うているのかが少し分かりにくいかなあ,ただこれがそういう面の理解も含めての演習ならすごくいいと思う
% % P6: 前提知識が少し多い気がする,何を聞かれているのかが分かるまでに時間がかかる.
% % P7: 色々役立つtipsみたいなのが散らばってていいと思うが,何を聞かれているのかが少し分かりにくいと感じた.

% \myquote{Many advance knowledges are required. Difficult to understand what the exercise says.}{PA6}




% \newpage

% \subsection{Experiment Design}

% \subsubsection{Task}

% Dataset:
% \begin{itemize}
%   \item Quizzes created from the \XXX system
%   \item Quizzes created from Sanfoundry
% \end{itemize}

% Test:
% \begin{itemize}
%   \item 3 Tests (Test A, B, and C)
%   \item Each test has 20 Python quizzes created from the \XXX system.
% \end{itemize}

% Task:
% \begin{itemize}
%   \item Both groups first take Test A.
%   \item First learning for 3 days. (Group A: Quiz-GitHub, Group B: Quiz-Sanfoundry). Notice that participants have to answer more than 50 quizzes.
%   \item Both groups then take Test B.
%   \item Second learning for 3days (Group A: Quiz-Sanfoundry, Group B: Quiz-GitHub). Notice that participants have to answer more than 50 quizzes.
%   \item Both groups take Test C.
% \end{itemize}

% Compensation:
% \begin{itemize}
% 	\item 5,000 yen and additional rewards according to a number of answered quizzes
% \end{itemize}


% \subsubsection{Participants}

% We set the following criteria for participants: 1) they must be junior or senior; 2) they must have taken at least one computer science course; 3) they had not joined software development.
% Our participants thus would be students who have learned programming in a course but do not have joined software development in a real world.
% We recruited 10 students for our evaluation (\shibato{}{P1--10; 9 male and 1 female}).


% \subsubsection{Procedure}

% Our participants first were asked to come to our laboratory for this study.
% After they signed a consent form, we explained the \XXX system and tasks, and gave time for practice.
% The participants then took \textit{Test 1}.


% \subsection{Results}

% \subsubsection{3 Tests}
% We conducted multi-level linear regression analysis to quantitatively examine the contribution of \XXX.


% \subsubsection{Questionnaire}


% \subsubsection{Time and Location}


% \subsubsection{Final Questionnaire}
% TBD

















